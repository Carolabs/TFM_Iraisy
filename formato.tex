\usepackage{blindtext}
%\usepackage[latin1]{inputenc}
\usepackage[spanish,es-lcroman,es-nosectiondot]{babel}
%------------------------------------------------------
\usepackage{array,amssymb,amsthm,amsmath,amstext}
\usepackage[font=small,bf]{caption}% Formato del caption
\usepackage{colortbl}% permite colorear tablas
\usepackage{fancyhdr}
\usepackage{fancyvrb}
\usepackage{graphicx}
\usepackage{pdfpages}
\usepackage{setspace,subfigure}
\usepackage{titlesec}
\usepackage{xcolor}
\usepackage[colorlinks=true,linkcolor=black,citecolor=magenta]{hyperref}
% Fuente ----------------------------------------------------------
\usepackage{helvet}
\renewcommand*\familydefault{\sfdefault} 
%------------------------------------------------------------------
% Comandos personales
\spanishdecimal{.}
\renewcommand{\spanishtablename}{Tabla}
\newcommand\crule[3][black]{\textcolor{#1}{\rule{#2}{#3}}}
\allowdisplaybreaks
\renewcommand{\refname}{Referencias}


% Formato hoja----------------------------------------------
\renewcommand{\baselinestretch}{1.25}% interlineado
\headsep 15.5mm      \topmargin -2cm    \textheight 24.5cm     \textwidth 16cm    \oddsidemargin 0.04 cm      \evensidemargin -0.5cm
\footnotesep=20pt   \footskip=40pt
% Formato de las cabaceras de página -----------------------------------------------------------------
\usepackage{fancyhdr}
\pagestyle{fancy}
\fancyhf{} % borrar todos los ajustes
\fancyfoot[C]{\textbf{\scriptsize\thepage}}
% Modifica el ancho de las líneas de cabecera y pie
\renewcommand{\headrulewidth}{0pt}
%\renewcommand{\footrulewidth}{0.4pt}
\renewcommand{\footrule}{\hrule height 0.4pt \vspace{4mm}}
%----------------------------------------------------------------------------------------------------------------
\usepackage{chngcntr}
\numberwithin{equation}{section}
\counterwithin{table}{section}% Numera las tablas por secciones
\counterwithin{figure}{section}% Numera las figuras por secciones
%--------------------------------------------------------------------
\setcounter{lofdepth}{2}% Introduce las subfiguras en lof
\allowdisplaybreaks
%%%%%%%%%%%%%%%%%%%%%%%%%%%%%%%%%%%%%%%%%%%%%%%%%%%%%%%%%%%%%%%%%%%%%%%%%%%%%%%%%%%%%%%%%%%
% Incluye la bibliografía como sección
\makeatletter
\renewenvironment{thebibliography}[1]
     {\section{\refname}% esta línea cambia la bibliografía a la categoría sección
      \@mkboth{\MakeUppercase\bibname}{\MakeUppercase\bibname}
      \list{\@biblabel{\@arabic\c@enumiv}}%
           {\settowidth\labelwidth{\@biblabel{#1}}
            \leftmargin\labelwidth
            \advance\leftmargin\labelsep
            \@openbib@code
            \usecounter{enumiv}
            \let\p@enumiv\@empty
            \renewcommand\theenumiv{\@arabic\c@enumiv}}
      \sloppy
      \clubpenalty4000
      \@clubpenalty \clubpenalty
      \widowpenalty4000%
      \sfcode`\.\@m}
     {\def\@noitemerr
       {\@latex@warning{Empty `thebibliography' environment}}
      \endlist}
\makeatother
%%%%%%%%%%%%%%%%%%%%%%%%%%%%%%%%%%%%%%%%%%%%%%%%%%%%%%%%%%%%%%%%%%%%%%%%%%%%%%%%%%%%%%%%%%%%%%%%%%%%%%%%%%%%%%%%%%%%%%%%%%%%%%%%%%%%%%%%%%%%%%%%%%%%%%%%%%%%%%%%%%%%%

\renewcommand\thesection{\arabic{section}}