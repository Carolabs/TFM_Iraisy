% !TeX spellcheck = es_ES

%__________________________ CAPÍTULO
\section{Estado del arte}
\label{ch:estado}

\commentIF{¿Cómo estructuro esta parte?}

\commentFG{Tu TFM necesita dos estados del arte distintos: uno para co-simulación y otro para IA.}

\commentFG{En la parte de co-simulación creo que será importante distinguir co-simulación implícita y explícita (no iterativa). Este TFM trata sobre todo de co-sim explícita; aquí será necesario explicar bien las soluciones que se han presentado en la literatura. Tienes un buen resumen en nuestro artículo de Multibody System Dynamics de 2022. Simplificando mucho, hay dos enfoques habituales: predicción de entradas (por ejemplo, a través de métodos de extrapolación) y correción de entradas (métodos que modifican las variables que se intercambian para eliminar los errores de la interfaz). Puedes mencionar también que otros métodos modifican el tamaño del paso de tiempo de comunicación, o que definen algoritmos de control durante una fase de análisis previa a la ejecución del código. Hasta la fecha, nuestro grupo no les ha prestado mucha atención, porque no es fácil usar estos métodos en tiempo real, pero es verdad que existen y son legítimos.}

\commentFG{Puedes mencionar que no existen muchas aplicaciones de la IA a co-sim, lo que hace este TFM novedoso e interesante para la investigación.}

\commentFG{Y luego necesitarás un estado del arte de IA. Como es un tema muy amplio, lo mejor es que el otro director te dé orientaciones sobre cómo avanzar con esto.}


%__________________________ SECCIÓN

